% Appendix

\section{Dataset and Split Details}
\label{app:data}

\subsection{ScienceWorld Task Types}

Table~\ref{tab:sw_tasks} lists the ScienceWorld task types used for training and out-of-distribution evaluation.

\begin{table}[h]
\centering
\small
\resizebox{\linewidth}{!}{%
\begin{tabular}{ll}
\toprule
\textbf{Split} & \textbf{Task Types} \\
\midrule
\multirow{4}{*}{Train (13)} & boil, melt, chemistry-mix, find-animal, \\
& find-plant, grow-plant, identify-life-stages-1, \\
& lifespan-longest-lived, inclined-plane-angle, \\
& measure-melting-point, power-component, \\
& test-conductivity, mendelian-genetics \\
\midrule
\multirow{4}{*}{OOD Test (12)} & freeze, change-state-of-matter, \\
& chemistry-mix-paint, find-non-living, \\
& grow-fruit, identify-life-stages-2, \\
& lifespan-shortest, inclined-plane-friction, \\
& use-thermometer, power-renewable, \\
& test-conductivity-unknown, genetics-unknown \\
\bottomrule
\end{tabular}
}
\caption{ScienceWorld task type splits.}
\label{tab:sw_tasks}
\end{table}


For each task type, we sample up to 30 variations (to bound collection time). Variations are split 60\%/20\%/20\% into train/validation/test using a fixed random seed (42).

\subsection{HLE Subject Categories}

\begin{table}[h]
\centering
\small
\begin{tabular}{lc}
\toprule
\textbf{Category} & \textbf{N (Test)} \\
\midrule
\multicolumn{2}{c}{\textit{In-Distribution (Training)}} \\
\midrule
Math & 200 \\
Physics & 46 \\
Chemistry & 23 \\
Biology/Medicine & 37 \\
Engineering & 16 \\
Computer Science/AI & 37 \\
\midrule
\multicolumn{2}{c}{\textit{Out-of-Distribution}} \\
\midrule
Humanities/Social Science & 193 \\
Other & 176 \\
\bottomrule
\end{tabular}
\caption{HLE test set distribution by category.}
\label{tab:hle_categories}
\end{table}


\section{Error Detection Rules}
\label{app:error_rules}

Table~\ref{tab:error_rules_detail} provides the full specification of error detection rules used to compute annealed error costs (AEC) during training.
Each rule consists of pattern strings matched against environment observations, a severity level (high/medium/low), and a description.

\paragraph{Design Principles.}
For HLE, we distinguish between model errors (format violations, Python exceptions) and external failures (HTTP errors, paywalls).
Model errors receive higher severity since they reflect controllable mistakes; external failures are marked low severity as they depend on third-party services.
For ScienceWorld, we only penalize truly invalid actions (commands not recognized by the environment parser).
Environmental feedback such as ``the door is not open'' or ``the object is already in your inventory'' represents normal exploration and is not treated as an error.

\paragraph{Severity Coefficients.}
Severity levels map to penalty coefficients in AEC: high (2.0), medium (1.0), low (0.25).
The warmup schedule uses $p_0{=}0.3$, $p_1{=}0.7$, $w_{\min}{=}0.3$, and $w_{\max}{=}1.0$.
These coefficients modulate the base penalty $1/N$ where $N$ is the expected episode length for the task category.

\begin{table*}[!htbp]
\centering
\small
\begin{tabular}{lllp{7cm}}
\toprule
\textbf{Category} & \textbf{Rule Name} & \textbf{Severity} & \textbf{Description} \\
\midrule
\multicolumn{4}{l}{\textit{\textbf{HLE Benchmark}}} \\
\midrule
\multirow{4}{*}{Format Errors}
 & format\_error & medium & Tool call format error---model did not follow required format \\
 & tool\_invalid\_args & medium & Invalid tool arguments or missing required parameters \\
 & tool\_parse\_error & medium & Tool call parsing failure \\
 & tool\_unknown & high & Model called a non-existent tool name \\
\midrule
\multirow{11}{*}{\shortstack[l]{Python\\Execution\\Errors}}
 & python\_traceback & high & Python execution exception with traceback \\
 & python\_name\_error & high & Undefined variable reference \\
 & python\_syntax\_error & high & Python syntax error \\
 & python\_indentation\_error & high & Python indentation error \\
 & python\_type\_error & high & Type mismatch error \\
 & python\_value\_error & medium & Invalid value error \\
 & python\_index\_error & medium & Index out of bounds \\
 & python\_key\_error & medium & Dictionary key not found \\
 & python\_attribute\_error & medium & Attribute access error \\
 & python\_import\_error & medium & Module import failure \\
 & python\_zero\_division & medium & Division by zero \\
 & python\_timeout & high & Code execution timeout \\
\midrule
\multirow{3}{*}{\shortstack[l]{Search Tool\\Errors}}
 & search\_no\_results & high & Search returned no results \\
 & search\_http\_error & low & HTTP connection error during search \\
 & search\_rate\_limit & low & Search rate limit exceeded \\
\midrule
\multirow{5}{*}{\shortstack[l]{Browse Tool\\Errors}}
 & browse\_403 & low & HTTP 403 Forbidden \\
 & browse\_404 & low & HTTP 404 Not Found \\
 & browse\_access\_denied & low & Access denied to resource \\
 & browse\_paywall & low & Paywall or subscription block \\
 & browse\_timeout & low & Browse request timeout \\
\midrule
\multicolumn{4}{l}{\textit{\textbf{ScienceWorld Benchmark}}} \\
\midrule
Invalid Action & no\_known\_action & high & Invalid action command not recognized by environment \\
\bottomrule
\end{tabular}
\caption{Detailed error detection rules used for annealed error cost (AEC) computation. Severity levels determine penalty coefficients: high (1.0), medium (0.8), low (0.2). Browse tool errors are marked low severity as they reflect external failures rather than model errors.}
\label{tab:error_rules_detail}
\end{table*}


\section{Training Dynamics}
\label{app:training}

\begin{table}[h]
\centering
\small
\begin{tabular}{lcc}
\toprule
\textbf{Model} & \textbf{Score $R^2$} & \textbf{Cost $R^2$} \\
\midrule
GPT-5 & 0.792 & 0.645 \\
DeepSeek-V3.2 & 0.731 & 0.618 \\
MiniMax-M2 & 0.718 & 0.587 \\
Kimi-K2 & 0.744 & 0.592 \\
Gemini-Flash & 0.689 & 0.551 \\
GPT-OSS-120B & 0.755 & 0.708 \\
\midrule
\textbf{Average} & 0.738 & 0.617 \\
\bottomrule
\end{tabular}
\caption{Per-model prediction $R^2$ on validation set.}
\label{tab:r2_scores}
\end{table}


The value model achieves strong predictive performance, with score prediction $R^2$ ranging from 0.689 (Gemini) to 0.792 (GPT-5). Cost prediction is slightly lower, likely due to higher variance in remaining cost depending on success/failure paths.

\section{Additional Experiments}
\label{app:more_experiments}

\paragraph{Learned model embeddings.}
\begin{figure}[t]
\centering
\includegraphics[width=\columnwidth]{figures/model_embeddings.png}
\caption{t-SNE visualization of learned model embeddings from the model encoder. The embeddings separate models by identity and form a clear cost-tier structure, with low-cost models (e.g., GPT-OSS, Gemini) distinct from higher-cost frontier models (e.g., GPT-5).}
\label{fig:model_embeddings}
\end{figure}

Figure~\ref{fig:model_embeddings} visualizes the learned model embeddings after training.
The encoder learns to distinguish the candidate models and organizes them by cost tier, suggesting it captures meaningful capability--cost structure beyond raw attributes.

\section{OpenRouter Baseline Details}
\label{app:openrouter}

\paragraph{Motivation.}
OpenRouter \citep{openrouter} provides an automatic routing API commonly used in commercial deployments.
We include OpenRouter as a representative commercial router baseline to contextualize \method against an off-the-shelf production routing system.

\paragraph{Model pool.}
Unlike our setting, which restricts routing to a fixed 6-model candidate pool (Table~\ref{tab:model_pool}), OpenRouter's automatic routing can choose from a much broader set of models.
In our evaluation, the OpenRouter baseline routes over the following pool (as reported by the API at routing time):

\begin{table*}[t]
\centering
\footnotesize
\begin{tabular}{ll}
\toprule
\textbf{Provider/Model} & \textbf{Provider/Model} \\
\midrule
\texttt{openai/gpt-5.1} & \texttt{openai/gpt-5} \\
\texttt{openai/gpt-5-mini} & \texttt{openai/gpt-5-nano} \\
\texttt{openai/gpt-4.1} & \texttt{openai/gpt-4.1-mini} \\
\texttt{openai/gpt-4.1-nano} & \texttt{openai/gpt-4o} \\
\texttt{openai/gpt-4o-2024-05-13} & \texttt{openai/gpt-4o-2024-08-06} \\
\texttt{openai/gpt-4o-2024-11-20} & \texttt{openai/gpt-4o-mini} \\
\texttt{openai/gpt-4o-mini-2024-07-18} & \texttt{openai/gpt-4-turbo} \\
\texttt{openai/gpt-4-turbo-preview} & \texttt{openai/gpt-4-1106-preview} \\
\texttt{openai/gpt-4} & \texttt{openai/gpt-3.5-turbo} \\
\texttt{openai/gpt-oss-120b} & \texttt{anthropic/claude-opus-4.5} \\
\texttt{anthropic/claude-opus-4.1} & \texttt{anthropic/claude-opus-4} \\
\texttt{anthropic/claude-sonnet-4.5} & \texttt{anthropic/claude-sonnet-4} \\
\texttt{anthropic/claude-3.7-sonnet} & \texttt{anthropic/claude-haiku-4.5} \\
\texttt{anthropic/claude-3.5-haiku} & \texttt{anthropic/claude-3-haiku} \\
\texttt{google/gemini-3-pro-preview} & \texttt{google/gemini-2.5-pro} \\
\texttt{google/gemini-2.0-flash-001} & \texttt{google/gemini-2.5-flash} \\
\texttt{mistralai/mistral-large} & \texttt{mistralai/mistral-large-2407} \\
\texttt{mistralai/mistral-large-2411} & \texttt{mistralai/mistral-medium-3.1} \\
\texttt{mistralai/mistral-nemo} & \texttt{mistralai/mistral-7b-instruct} \\
\texttt{mistralai/mixtral-8x7b-instruct} & \texttt{mistralai/mixtral-8x22b-instruct} \\
\texttt{mistralai/codestral-2508} & \texttt{x-ai/grok-4} \\
\texttt{x-ai/grok-3} & \texttt{x-ai/grok-3-mini} \\
\texttt{deepseek/deepseek-r1} & \texttt{meta-llama/llama-3.3-70b-instruct} \\
\texttt{meta-llama/llama-3.1-405b-instruct} & \texttt{meta-llama/llama-3.1-70b-instruct} \\
\texttt{meta-llama/llama-3.1-8b-instruct} & \texttt{meta-llama/llama-3-70b-instruct} \\
\texttt{meta-llama/llama-3-8b-instruct} & \texttt{qwen/qwen3-235b-a22b} \\
\texttt{qwen/qwen3-32b} & \texttt{qwen/qwen3-14b} \\
\texttt{cohere/command-r-plus-08-2024} & \texttt{cohere/command-r-08-2024} \\
\texttt{moonshotai/kimi-k2-thinking} & \texttt{perplexity/sonar} \\
\bottomrule
\end{tabular}
\caption{OpenRouter automatic routing model pool used by our OpenRouter baseline (as reported by the API at routing time).}
\label{tab:openrouter_pool}
\end{table*}


\paragraph{Implications for comparison.}
Because OpenRouter can select from a superset of models (including multiple frontier and provider-specific options not present in our pool), it represents a stronger routing setting than ours.
We still include it as a practical reference point, but comparisons should be interpreted with this mismatch in candidate pools in mind.

\paragraph{Why does OpenRouter perform poorly on ScienceWorld?}
Figure~\ref{fig:openrouter_model_usage} shows the model usage distribution of OpenRouter on both benchmarks.
On ScienceWorld, OpenRouter predominantly selects lightweight models: Mistral-Nemo accounts for 68\% of all model calls, followed by GPT-5-nano (24\%) and Claude-4.5 (7\%).
These models, while cost-efficient, lack the reasoning capability required for ScienceWorld's procedural scientific tasks.
In contrast, on HLE, OpenRouter allocates 54\% of calls to Claude-4.5---a much stronger frontier model---along with Sonar (23\%) and Mistral-Nemo (15\%), reflecting a more appropriate difficulty assessment for that benchmark.

This discrepancy reveals a key limitation of general-purpose commercial routers: without task-specific training, they may underestimate the difficulty of unfamiliar domains and over-rely on cheaper models.
ScienceWorld's text-based interface and seemingly simple commands may mislead the router into treating it as an easy task, when in fact it requires multi-step planning and precise action sequencing that weaker models struggle to execute.
The resulting negative scores ($-26.4$ on Test, $-26.9$ on OOD) indicate that the selected models frequently fail to make meaningful progress toward task completion, as judged by the environment's scoring function.

\begin{figure}[!htbp]
\centering
\includegraphics[width=\columnwidth]{figures/openrouter_model_usage.pdf}
\caption{Model usage distribution of OpenRouter on ScienceWorld and HLE. On ScienceWorld, OpenRouter predominantly selects lightweight models (Mistral-Nemo: 68\%, GPT-5-nano: 24\%), while on HLE it primarily uses stronger models (Claude-4.5: 54\%), reflecting different difficulty assessments.}
\label{fig:openrouter_model_usage}
\end{figure}

\section{Reproducibility}
\label{app:reproduce}

\paragraph{Code and Data.}
All code, trained models, and trajectory data will be released at \url{[anonymized for review]}. The framework is implemented in Python 3.10 with PyTorch 2.0.

\paragraph{Compute Requirements.}
\begin{itemize}
\item Trajectory collection: 6$\times$ A100 GPUs, approximately 48 hours
\item Embedding precomputation: 1$\times$ A100, approximately 4 hours
\item Router training: 1$\times$ A100, approximately 2 hours
\item Evaluation: 2$\times$ A100, approximately 8 hours per benchmark
\end{itemize}

\paragraph{API Costs.}
Total API cost for experiments: approximately \$3,500 USD across data collection and evaluation.

\section{Prompts and Tool Schemas}
\label{app:prompts}

We report the prompts used in our evaluation setup. For brevity, we omit the format-error correction prompts.

\subsection{ScienceWorld Agent Prompt}
\label{app:prompts:sw}
\begin{lstlisting}[style=prompt]
You are a helpful assistant interacting with a text-based science simulation environment.
Your goal is to complete science experiments by issuing text commands.

## How to Interact
You issue commands by writing them in a ```text``` code block. The environment will respond with observations.

<format_example>
THOUGHT: I should explore my surroundings first.

```text
look around
```
</format_example>

## Available Command Types
Common commands include:
- Movement: "go to [location]", "open door", "go through door"
- Interaction: "pick up [object]", "put [object] in [container]", "activate [object]"
- Observation: "look around", "look at [object]", "inventory"
- Task-specific: "focus on [object]", "use [tool] on [object]"

## Query Commands (Free Actions)
You can query available actions without consuming a game turn:
- `?navigation` - Show movement actions (go, walk, move)
- `?object` - Show object manipulation (pick up, put, pour)
- `?observation` - Show observation actions (look, examine, inventory)
- `?device` - Show device control (activate, turn on/off, use)
- `?door` - Show door/container actions (open, close)
- `?electrical` - Show electrical actions (connect, disconnect)
- `?interaction` - Show interaction actions (mix, eat, focus)
- `?all` - Show all valid actions
- `?categories` - Show query help

Use queries to explore available actions before deciding your next move.

## Important Notes
- Issue ONE command per response
- Include a THOUGHT section explaining your reasoning
- The environment is turn-based - wait for observations before issuing the next command
- Some tasks require multiple steps to complete
\end{lstlisting}

\begin{lstlisting}[style=prompt]
<task_description>
{{task_description}}
</task_description>

<initial_observation>
{{initial_observation}}
</initial_observation>

<instructions>
Complete the science experiment described above. You are interacting with a simulated environment.
Issue commands one at a time and observe the results.

**Tip:** Use query commands like `?navigation` or `?object` to explore available actions without consuming a turn.

**Strategy hints:**
- First explore: use "open door to [room]" then "go to [room]" to navigate
- Look around each room to find objects you need
- Pick up objects with "pick up [object]"
- For heating: find a stove, turn it on with "activate [stove]", place container on it
- For cooling: use a freezer or refrigerator
- Use "focus on [object]" to examine substances

To complete the task, perform the necessary actions described in the task description.
When you believe the task is complete, issue the command:

```text
task completed
```

Remember:
- Think step by step about what actions are needed
- Explore your environment to find needed objects
- Some actions may require prerequisites (e.g., picking up an object before using it)
</instructions>
\end{lstlisting}


\subsection{HLE Agent Prompt}
\label{app:prompts:hle}
The HLE system prompt injects tool schemas via a template variable. We list each tool schema explicitly in Appendix~\ref{app:tool_schemas}.
\begin{lstlisting}[style=prompt]
You are an expert problem solver tackling challenging questions from Humanity's Last Exam.
These questions require deep reasoning, careful research, and precise answers.

## Available Tools

<tools>

{{ tool.to_xml() }}

</tools>

## Tool Call Formats

You can use either XML or JSON format inside <tool_call> tags:

**Format A: XML**
<tool_call>
<tool_name>
  <param1>value1</param1>
  <param2>value2</param2>
</tool_name>
</tool_call>

**Format B: JSON**
<tool_call>{"name": "tool_name", "arguments": {"param1": "value1"}}</tool_call>

## Examples

<tool_call>
<search>
  <query>maximum likelihood estimation</query>
</search>
</tool_call>

<tool_call>{"name": "python", "arguments": {"code": "print(2 + 2)"}}</tool_call>

## Strategy for HLE Questions

1. **Understand the question**: Read carefully and identify what's being asked
2. **Break down the problem**: Decompose into sub-problems if needed
3. **Research if needed**: Use `search` for factual information you're unsure about
4. **Verify sources**: Use `browse` to read primary sources when accuracy matters
5. **Compute when needed**: Use `python` for calculations, data analysis
6. **Synthesize**: Combine information from multiple sources
7. **Verify your answer**: Double-check before submitting
8. **Submit**: Use `answer` tool with your final answer

## Important Notes

- These are challenging questions - take your time
- Be precise - exact answers are often required
- Show your reasoning before using tools
- If uncertain, express confidence level in your answer
\end{lstlisting}

\begin{paperbox}[breakable]{HLE Instance Prompt}
\textbf{HLE Question} \\
\texttt{\{\% if subject \%\}} \\
\textbf{Subject}: \texttt{\{\{ subject \}\}} \\
\texttt{\{\% endif \%\}}

\vspace{0.2cm}
\texttt{\{\{ question | default(task) \}\}}

\vspace{0.2cm}
\hrule
\vspace{0.2cm}

Solve this step by step. Use tools to research and verify.
Submit your final answer using the \texttt{answer} tool:
\texttt{<tool\_call>\{"name": "answer", "arguments": \{"answer": "your final answer"\}\}</tool\_call>}
\end{paperbox}


\subsection{Tool Schemas}
\label{app:tool_schemas}
We list each tool's JSON schema (name, description, and parameter schema) used by the HLE agent.
\begin{figure*}[!htbp]
\begin{tcolorbox}[
    enhanced,
    title=\textbf{Tool Schema: search},
    colback=white,
    colframe=black!90,
    coltitle=white,
    fonttitle=\bfseries\small,
    boxrule=0.8pt,
    arc=2mm,
    attach boxed title to top left={xshift=4mm, yshift=-3mm},
    boxed title style={colback=black!90, sharp corners=south}
]
\begin{lstlisting}[language=json]
{
  "name": "search",
  "description": "Search the web for information using Google",
  "parameters": {
    "type": "object",
    "properties": {
      "query": {
        "oneOf": [
          { "type": "string" },
          { "type": "array", "items": { "type": "string" } }
        ],
        "description": "Search query or list of queries for batch search"
      },
      "num_results": {
        "type": "integer",
        "description": "Number of results to return (default: 10)",
        "minimum": 1,
        "maximum": 100
      }
    }
  },
  "required": [ "query" ]
}
\end{lstlisting}
\end{tcolorbox}
\end{figure*}

\begin{figure*}[!htbp]
\begin{tcolorbox}[
    enhanced,
    title=\textbf{Tool Schema: browse},
    colback=white,
    colframe=black!90,
    coltitle=white,
    fonttitle=\bfseries\small,
    boxrule=0.8pt,
    arc=2mm,
    attach boxed title to top left={xshift=4mm, yshift=-3mm},
    boxed title style={colback=black!90, sharp corners=south}
]
\begin{lstlisting}[language=json]
{
  "name": "browse",
  "description": "Visit webpage(s) and extract information based on a goal",
  "parameters": {
    "type": "object",
    "properties": {
      "url": {
        "type": "string",
        "description": "URL to visit. Can be a single URL or array of URLs."
      },
      "goal": {
        "type": "string",
        "description": "The goal of the visit - what information to extract from the webpage(s)."
      },
      "extract_mode": {
        "type": "string",
        "enum": [ "text", "markdown", "html" ],
        "description": "Content extraction mode (default: text). Only used when LLM summary is disabled."
      }
    }
  },
  "required": [ "url", "goal" ]
}
\end{lstlisting}
\end{tcolorbox}
\end{figure*}

\begin{tcolorbox}[
    enhanced,
    title=\textbf{Tool Schema: python},
    colback=white,
    colframe=black!90,
    coltitle=white,
    fonttitle=\bfseries\small,
    boxrule=0.8pt,
    arc=2mm,
    attach boxed title to top left={xshift=4mm, yshift=-3mm},
    boxed title style={colback=black!90, sharp corners=south}
]
\begin{lstlisting}[language=json]
{
  "name": "python",
  "description": "Execute Python code and return the output",
  "parameters": {
    "type": "object",
    "properties": {
      "code": {
        "type": "string",
        "description": "Python code to execute"
      }
    }
  },
  "required": [ "code" ]
}
\end{lstlisting}
\end{tcolorbox}

\begin{figure*}[!htbp]
\begin{tcolorbox}[
    enhanced,
    title=\textbf{Tool Schema: answer},
    colback=white,
    colframe=black!90,
    coltitle=white,
    fonttitle=\bfseries\small,
    boxrule=0.8pt,
    arc=2mm,
    attach boxed title to top left={xshift=4mm, yshift=-3mm},
    boxed title style={colback=black!90, sharp corners=south}
]
\begin{lstlisting}[language=json]
{
  "name": "answer",
  "description": "Submit your final answer to complete the task",
  "parameters": {
    "type": "object",
    "properties": {
      "answer": {
        "type": "string",
        "description": "The final answer to submit"
      },
      "confidence": {
        "type": "number",
        "description": "Confidence level from 0 to 1 (optional)",
        "minimum": 0,
        "maximum": 1
      },
      "reasoning": {
        "type": "string",
        "description": "Brief explanation of how you arrived at the answer (optional)"
      }
    }
  },
  "required": [ "answer" ]
}
\end{lstlisting}
\end{tcolorbox}
\end{figure*}


\subsection{Routing Model Prompt for LLM Router / Router-R1}
\label{app:prompts:routing}
We report the turn-level routing prompt used by the LLM Router baseline and by Router-R1. In both baselines, the routing model is queried at each turn to produce a \texttt{<select>} decision. They use the same prompt template; Router-R1 uses a trained \texttt{Qwen2.5-7B-Instruct} routing model, while LLM Router directly uses \texttt{DeepSeek-V3.2} (no training).
\begin{paperbox}[breakable]{LLM Router / Router-R1 System Prompt (Policy LLM)}
You are a model routing assistant. Your job is to select the best model for the given task.

\vspace{0.3cm}
\textbf{Available models:} \\
\texttt{\{model\_list\}}

\vspace{0.3cm}
\textbf{Instructions}
\begin{enumerate}[leftmargin=*, label=\arabic*.]
\item Analyze the task in \texttt{<think>...</think>} tags
\item Select the best model using \texttt{<select>model\_name</select>}
\end{enumerate}

\vspace{0.2cm}
\textbf{Example} \\
\texttt{<think>This is a simple math question. A cheaper model would suffice.</think>} \\
\texttt{<select>deepseek/deepseek-v3.2</select>}

\vspace{0.3cm}
\textbf{Rules}
\begin{itemize}
\item You MUST output exactly one \texttt{<select>} tag
\item The model name must match exactly from the available list
\item Consider: task complexity, model strengths, cost-effectiveness
\end{itemize}
\end{paperbox}

\begin{figure*}[!htbp]
\begin{paperbox}{LLM Router / Router-R1 Model Descriptors}
\texttt{openai/gpt-5}: Cost 1.25/10. 400K context. Top performer with SW 48\% SR and HLE 25\% SR. Best for complex multi-step reasoning and hard science problems. \\
\texttt{openai/gpt-oss-120b}: Cost 0.09/0.36. 131K context. Lowest cost with SW 33\% SR and HLE 10\% SR. Excellent cost-efficiency for moderate difficulty tasks. \\
\texttt{moonshotai/kimi-k2-0905}: Cost 0.39/1.9. 131K context. SW 31\% SR and HLE 11\% SR. Balanced option for general-purpose tasks with good instruction following. \\
\texttt{deepseek/deepseek-v3.2}: Cost 0.27/0.42. 164K context. SW 16\% SR and HLE 16\% SR. Strong on math and coding. Good for structured reasoning tasks. \\
\texttt{google/gemini-2.5-flash-lite}: Cost 0.1/0.4. 1M context. SW 23\% SR and HLE 6\% SR. Largest context window. Best for long document analysis. \\
\texttt{minimax/minimax-m2}: Cost 0.2/1. 196K context. SW 22\% SR and HLE 8\% SR. Fast inference. Suitable for simple QA and straightforward tasks.
\end{paperbox}
\end{figure*}


\subsection{Browse Extractor Prompt}
\label{app:prompts:browse}
The HLE \texttt{browse} tool can optionally use an LLM to extract and summarize relevant evidence from retrieved webpages. We report the extractor prompt used for this LLM-based summarization.
\begin{figure*}[!htbp]
\begin{paperbox}{Browse Extractor Prompt (LLM Summary)}
Please process the following webpage content and user goal to extract relevant information:

\vspace{0.3cm}
\textbf{\large Webpage Content} \\
\{webpage\_content\}

\vspace{0.3cm}
\textbf{\large User Goal} \\
\{goal\}

\vspace{0.3cm}
\textbf{\large Task Guidelines}
\begin{enumerate}[leftmargin=*, label=\arabic*.]
\item \textbf{Content Scanning for Rationale}: Locate the \textbf{specific sections/data} directly related to the user's goal within the webpage content
\item \textbf{Key Extraction for Evidence}: Identify and extract the \textbf{most relevant information} from the content; do not miss important information. Output the \textbf{full original context} as much as possible (it can exceed three paragraphs).
\item \textbf{Summary Output for Summary}: Organize into a concise paragraph with logical flow, prioritizing clarity and judging the contribution of the information to the goal.
\end{enumerate}

\vspace{0.2cm}
\textbf{Output Format}: JSON format containing \texttt{"rational"}, \texttt{"evidence"}, and \texttt{"summary"} fields.
\end{paperbox}
\end{figure*}


\subsection{HLE Judge Prompt}
\label{app:prompts:judge}
HLE scoring uses an LLM-as-a-judge component aligned with the official HLE evaluation. We report the judge prompt used to compare a model response against the provided reference answer.
\begin{paperbox}[breakable]{HLE Judge Prompt}
Judge whether the following \texttt{[response]} to \texttt{[question]} is correct or not based on the precise and unambiguous \texttt{[correct\_answer]} below.

\vspace{0.3cm}
\texttt{[question]: \{question\}} \\
\texttt{[response]: \{response\}}

\vspace{0.3cm}
Your judgement must be in the format and criteria specified below:

\vspace{0.2cm}
\textbf{extracted\_final\_answer}: The final exact answer extracted from the \texttt{[response]}. Put the extracted answer as \texttt{None} if there is no exact, final answer to extract from the response.

\vspace{0.2cm}
\texttt{[correct\_answer]: \{correct\_answer\}}

\vspace{0.2cm}
\textbf{reasoning}: Explain why the extracted\_final\_answer is correct or incorrect based on \texttt{[correct\_answer]}, focusing only on whether there are meaningful differences. Do not comment on background, do not attempt to solve the problem, do not argue for any answer different than \texttt{[correct\_answer]}; focus only on whether the answers match.

\vspace{0.2cm}
\textbf{correct}: Answer \texttt{yes} if extracted\_final\_answer matches the \texttt{[correct\_answer]} given above, or is within a small margin of error for numerical problems. Answer \texttt{no} otherwise.

\vspace{0.2cm}
\textbf{confidence}: The extracted confidence score between 0\% and 100\% from \texttt{[response]}. Put 100 if there is no confidence score available.

\vspace{0.2cm}
Respond with a JSON object containing these four fields: extracted\_final\_answer, reasoning, correct, confidence.
\end{paperbox}

