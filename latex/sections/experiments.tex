% Experiments section

We evaluate \method on two challenging multi-turn benchmarks and analyze the learned routing patterns.

\subsection{Experimental Setup}

\paragraph{Benchmarks.}
We evaluate on two diverse benchmarks:

\textbf{ScienceWorld} \citep{wang2022scienceworld} is a text-based interactive environment requiring procedural scientific reasoning. Tasks include boiling water, growing plants, testing conductivity, and Mendelian genetics experiments. The environment provides a terminal score $S_{\text{final}} \in [0, 100]$ based on task completion. We use 13 task types for training/validation with a 60\%/20\%/20\% variation split, reserving 12 task types for out-of-distribution testing. Episodes are limited to 50 steps and \$2.0 cost.

\textbf{HLE (Humanity's Last Exam)} is a challenging long-context benchmark spanning academic domains. Questions require multi-step reasoning with tool use (web search, Python execution, file reading). Success is binary based on answer correctness. We use 6 subject categories (Math, Physics, Chemistry, Biology/Medicine, Engineering, Computer Science/AI) for training, with 2 categories (Humanities/Social Science, Other) held out for OOD evaluation.

\paragraph{Model Pool.}
We evaluate with 6 frontier LLMs spanning a 100$\times$ cost range:

\begin{table}[h]
\centering
\small
\begin{tabular}{lrrr}
\toprule
\textbf{Model} & \textbf{Context} & \textbf{In \$/M} & \textbf{Out \$/M} \\
\midrule
GPT-5 & 400K & 1.25 & 10.00 \\
DeepSeek-V3.2 & 164K & 0.27 & 0.42 \\
MiniMax-M2 & 197K & 0.20 & 1.00 \\
Kimi-K2 & 131K & 0.39 & 1.90 \\
Gemini-2.5-Flash-Lite & 1M & 0.10 & 0.40 \\
GPT-OSS-120B & 131K & 0.09 & 0.36 \\
\bottomrule
\end{tabular}
\caption{Model pool with pricing (per million tokens).}
\label{tab:model_pool}
\end{table}


\paragraph{Training Details.}
We train for 100 epochs with early stopping (patience=3) using AdamW optimizer (lr=$10^{-3}$, weight decay=0.01) and cosine annealing. Batch size is 64. History embeddings are precomputed using vLLM for efficiency. Training takes approximately 2 hours on a single A100 GPU.

\paragraph{Baselines.}
We compare against: (1) \textbf{Single-model baselines}: each model used exclusively; (2) \textbf{Roulette}: uniform random selection at each step; (3) \textbf{Router-R1}: a multi-turn LLM router baseline; (4) \textbf{LLM Router}: a multi-turn LLM router baseline; (5) \textbf{OpenRouter}: a representative commercial router \citep{openrouter} using OpenRouter's automatic routing API; (6) \textbf{Step-wise single-turn routers}: RouterDC \citep{chen2024routerdc}, EmbedLLM \citep{wang2024embedllm}, and Avengers \citep{lu2024avengers} applied independently at each step. We emphasize that OpenRouter can route over a substantially larger candidate set than our fixed 6-model pool (Appendix~\ref{app:openrouter}), so it is advantaged in terms of available model options.

\subsection{Main Results}
\label{sec:main_results}

\begin{table*}[t]
\centering
\small
\begin{tabular}{lccc ccc}
\toprule
& \multicolumn{3}{c}{\textbf{ScienceWorld (Test)}} & \multicolumn{3}{c}{\textbf{HLE (Test)}} \\
\cmidrule(lr){2-4}\cmidrule(lr){5-7}
\textbf{Method} & Avg Score$\uparrow$ & Total Cost (\$)$\downarrow$ & Avg Turns & Acc. (\%)$\uparrow$ & Total Cost (\$)$\downarrow$ & Avg Turns \\
\midrule
\multicolumn{7}{c}{\textit{Single-Model Baselines}} \\
\midrule
GPT-5 & \textbf{48.4} & \$13.91 & 29.2 & \textbf{25.1} & \$61.77 & 7.0 \\
DeepSeek-V3.2 & 13.1 & \$2.86 & 45.1 & 15.6 & \$22.40 & 26.0 \\
MiniMax-M2 & -0.5 & \$3.22 & 33.0 & 7.8 & \$18.06 & 27.0 \\
Kimi-K2 & 5.2 & \$2.54 & 31.8 & 11.4 & \$12.03 & 16.2 \\
Gemini-Flash & 4.2 & \$0.32 & 30.4 & 5.6 & \$2.97 & 6.8 \\
GPT-OSS-120B & 26.6 & \$0.49 & 32.8 & 9.7 & \$0.71 & 11.4 \\
\midrule
\multicolumn{7}{c}{\textit{Single-Turn Routers (episode-level, fixed model)}} \\
\midrule
RouterDC & -- & -- & -- & -- & -- & -- \\
EmbedLLM & -- & -- & -- & -- & -- & -- \\
Avengers & -- & -- & -- & -- & -- & -- \\
\midrule
\multicolumn{7}{c}{\textit{Multi-Turn Routers}} \\
\midrule
Roulette & 37.7 & \$3.94 & 28.5 & 24.2 & \$28.00 & 10.2 \\
LLM Router & 19.8 & \$12.19 & 36.6 & 24.0 & \$8.52 & 5.8 \\
Router-R1 & -- & -- & -- & -- & -- & -- \\
OpenRouter & -26.4 & \$3.03 & 38.2 & 18.3 & \$40.11 & 5.6 \\
\midrule
\rowcolor{green!10}
\method ($\lambda$=0) & -- & -- & -- & -- & -- & -- \\
\rowcolor{green!10}
\methodplus ($\lambda$=0) & -- & -- & -- & -- & -- & -- \\
\bottomrule
\end{tabular}
\caption{Main results on ScienceWorld and HLE test sets. Avg Score/Avg Turns are per-episode averages (ScienceWorld score in 0--100; HLE accuracy in \%), while Total Cost is summed over evaluated episodes. ``--'' indicates results omitted or not available.}
\label{tab:main_results}
\end{table*}


Table~\ref{tab:main_results} presents our main results. Key findings:

\paragraph{Matching best single-model at lower cost.}
On ScienceWorld, GPT-5 achieves the strongest single-model performance (48.4 score). We compare routing methods against this oracle baseline while tracking cost and steps.

\paragraph{Outperforming naive routing.}
Roulette achieves strong performance (37.7 score on ScienceWorld; 24.2\% on HLE) by mixing model capabilities. In contrast, the commercial OpenRouter baseline performs poorly on ScienceWorld.

\paragraph{Cross-benchmark transfer.}
On HLE, LLM Router achieves 24.0\% accuracy, close to GPT-5 (25.1\%), while OpenRouter attains 18.3\%.

\subsection{Out-of-Distribution Generalization}

\begin{table*}[t]
\centering
\small
\begin{tabular}{lccc ccc}
\toprule
& \multicolumn{3}{c}{\textbf{ScienceWorld (Held-out Task Types)}} & \multicolumn{3}{c}{\textbf{HLE (Held-out Subjects)}} \\
\cmidrule(lr){2-4}\cmidrule(lr){5-7}
\textbf{Method} & Avg Score$\textcolor{green!60!black}{\uparrow}$ & Total Cost (\$)$\textcolor{red!70!black}{\downarrow}$ & Avg Turns & Acc. (\%)$\textcolor{green!60!black}{\uparrow}$ & Total Cost (\$)$\textcolor{red!70!black}{\downarrow}$ & Avg Turns \\
\midrule
\multicolumn{7}{c}{\textit{Single-Model Baselines}} \\
\midrule
GPT-5 & 4.9 & \$47.58 & 19.3 & 34.81 & \$65.33 & 13.00 \\
DeepSeek-V3.2 & -4.2 & \$22.80 & 42.2 & 28.70 & \$22.24 & 24.60 \\
MiniMax-M2 & 0.9 & \$10.91 & 30.1 & 8.90 & \$9.75 & 27.80 \\
Kimi-K2 & 0.2 & \$8.92 & 29.9 & 20.10 & \$9.80 & 15.50 \\
Gemini-2.5-Flash-Lite & -2.1 & \$1.48 & 29.6 & 8.4 & \$2.06 & 6.0 \\
GPT-OSS-120B & 1.1 & \$4.20 & 32.2 & 11.4 & \$2.14 & 18.5 \\
\midrule
\multicolumn{7}{c}{\textit{Single-Turn Routers (episode-level)}} \\
\midrule
RouterDC$^{a}$ & 5.5 & \$2.48 & 27.4 & 17.9 & \$13.49 & 16.6 \\
EmbedLLM$^{b}$ & 5.0 & \$3.04 & 35.9 & 33.6 & \$56.83 & 15.4 \\
AvengersPro$^{c}$ & 2.4 & \$4.12 & 38.9 & 30.6 & \$33.26 & 19.6 \\
\midrule
\multicolumn{7}{c}{\textit{Multi-Turn Routers (turn-level)}} \\
\midrule
Random Router & -8.11 & \$20.32 & 31.62 & 23.8 & \$14.68 & 9.8 \\
LLM Router & -0.36 & \$28.25 & 19.5 & \textbf{36.00} & \$35.57 & 10.10 \\
Router-R1 & 2.1 & \$20.97 & 31.0 & 35.1 & \$60.73 & 9.1 \\
OpenRouter$^{\dagger}$ & -26.9 & \$15.45 & 38.1 & 34.0 & \$154.32 & 7.2 \\
\midrule
\rowcolor{blue!10}
\textbf{\method~(ours)} & \textbf{9.9} & \$16.27 & 22.1 & \textbf{38.57} & \$31.17 & 13.94 \\
\rowcolor{blue!6}
\multicolumn{1}{l}{\scriptsize\textit{\quad $\Delta$ vs GPT-5}} &
{\scriptsize\textcolor{blue!70!black}{\textbf{+5.0}}} &
{\scriptsize\textcolor{orange!85!black}{\textbf{\textit{saving 65.8\%}}}} &
{\scriptsize\textcolor{black!60}{--}} &
{\scriptsize\textcolor{blue!70!black}{\textbf{+3.8}}} &
{\scriptsize\textcolor{orange!85!black}{\textbf{\textit{saving 52.3\%}}}} &
{\scriptsize\textcolor{black!60}{--}} \\
\rowcolor{blue!6}
\multicolumn{1}{l}{\scriptsize\textit{\quad $\Delta$ vs Router-R1}} &
{\scriptsize\textcolor{blue!70!black}{\textbf{+7.8}}} &
{\scriptsize\textcolor{orange!85!black}{\textbf{\textit{saving 22.4\%}}}} &
{\scriptsize\textcolor{black!60}{--}} &
{\scriptsize\textcolor{blue!70!black}{\textbf{+3.5}}} &
{\scriptsize\textcolor{orange!85!black}{\textbf{\textit{saving 48.7\%}}}} &
{\scriptsize\textcolor{black!60}{--}} \\
\bottomrule
\end{tabular}
\caption{OOD results on held-out ScienceWorld task types and held-out HLE subject categories. Avg Score/Avg Turns are per-episode averages (ScienceWorld score in $[-100,100]$; HLE accuracy in \%), while Total Cost is summed over evaluated episodes. The $\Delta$ lines summarize \method's score gains and relative cost savings (percentage) compared to GPT-5 and Router-R1. $^{a}$RouterDC~\citep{chen2024routerdc}; $^{b}$EmbedLLM~\citep{wang2024embedllm}; $^{c}$AvengersPro (Avengers~\citep{zhang2025avengers}). $^{\dagger}$OpenRouter~\citep{openrouter} uses a fixed provider-side routing API and does not allow customizing the candidate model pool, so its pool differs from ours. ``--'' indicates results omitted or not available.}
\label{tab:ood_results}
\end{table*}


Table~\ref{tab:ood_results} reports OOD performance on held-out ScienceWorld task types and held-out HLE categories. We include OpenRouter as a strong practical baseline that is advantaged by a larger candidate pool (Appendix~\ref{app:openrouter}).

\subsection{Analysis}
\label{sec:analysis}

\begin{figure}[t]
\centering
\includegraphics[width=0.80\columnwidth]{figures/model_embeddings.png}
\caption{t-SNE visualization of learned model embeddings from the model encoder. The embeddings separate models by identity and form a clear cost-tier structure, with low-cost models (e.g., GPT-OSS, Gemini) distinct from higher-cost frontier models (e.g., GPT-5).}
\label{fig:model_embeddings}
\end{figure}

\paragraph{Learned model embeddings.}
Figure~\ref{fig:model_embeddings} visualizes the learned model embeddings after training.
The encoder learns to distinguish the candidate models and organizes them by cost tier, suggesting it captures meaningful capability--cost structure beyond raw attributes.

\begin{figure}[t]
\centering
\begin{tikzpicture}[scale=0.85]
\begin{axis}[
    ybar stacked,
    bar width=12pt,
    xlabel={Episode Phase},
    ylabel={Model Usage (\%)},
    symbolic x coords={Early (1-5), Mid (6-15), Late (16+)},
    xtick=data,
    ymin=0, ymax=100,
    legend style={at={(0.5,-0.25)}, anchor=north, legend columns=3, font=\tiny},
    enlarge x limits=0.25,
    width=0.95\columnwidth,
    height=5cm,
]
\addplot[fill=gpt5color!80] coordinates {(Early (1-5), 51.6) (Mid (6-15), 38.0) (Late (16+), 29.8)};
\addplot[fill=gptosscolor!80] coordinates {(Early (1-5), 19.0) (Mid (6-15), 24.0) (Late (16+), 28.1)};
\addplot[fill=geminicolor!80] coordinates {(Early (1-5), 10.0) (Mid (6-15), 14.0) (Late (16+), 15.8)};
\addplot[fill=deepseekcolor!80] coordinates {(Early (1-5), 7.7) (Mid (6-15), 10.0) (Late (16+), 8.8)};
\addplot[fill=kimicolor!80] coordinates {(Early (1-5), 6.3) (Mid (6-15), 8.0) (Late (16+), 10.5)};
\addplot[fill=minimaxcolor!80] coordinates {(Early (1-5), 5.4) (Mid (6-15), 6.0) (Late (16+), 7.0)};
\legend{GPT-5, GPT-OSS, Gemini, DeepSeek, Kimi, MiniMax}
\end{axis}
\end{tikzpicture}
\caption{Model usage distribution across episode phases on ScienceWorld. The router prefers GPT-5 (expensive) for early planning and shifts to cheaper models in later phases.}
\label{fig:phase_usage}
\end{figure}

\paragraph{Phase-dependent model selection.}
Figure~\ref{fig:phase_usage} shows model usage across episode phases. In early steps (1-5), the router heavily favors GPT-5 (51.6\%), which excels at initial planning. As episodes progress, usage shifts toward cheaper models: GPT-OSS increases from 19.0\% to 28.1\%, while GPT-5 decreases from 51.6\% to 29.8\%. This pattern reflects the insight that early decisions are more critical for task success.

\paragraph{Error-triggered model switching.}
We analyze model switches triggered by errors (format errors, invalid actions). On ScienceWorld, 4.1\% of switches follow errors, with 100\% of error-to-premium switches leading to eventual success. On HLE, 49.0\% of switches are error-triggered, with format errors being most common (233 instances). The router learns to ``escalate'' to stronger models when errors occur.

\begin{table}[t]
\centering
\small
\begin{tabular}{llc}
\toprule
\textbf{Model} & \textbf{Task} & \textbf{Spec. Index} \\
\midrule
\rowcolor{geminicolor!20} Gemini-Flash & chemistry-mix & 7.02$\times$ \\
\rowcolor{geminicolor!20} Gemini-Flash & identify-life & 4.06$\times$ \\
\rowcolor{gptosscolor!20} GPT-OSS-120B & melt & 3.35$\times$ \\
\rowcolor{gpt5color!20} GPT-5 & boil & 3.02$\times$ \\
\rowcolor{gpt5color!20} GPT-5 & grow-plant & 2.53$\times$ \\
\rowcolor{gptosscolor!20} GPT-OSS-120B & lifespan & 1.90$\times$ \\
\bottomrule
\end{tabular}
\caption{Model specialization across ScienceWorld task categories. Specialization index = model's usage fraction on task / overall usage fraction. Values $>$1 indicate the model is used more often for that task than average.}
\label{tab:specialization}
\end{table}


\paragraph{Task-specific model specialization.}
Table~\ref{tab:specialization} reveals emergent specialization: Gemini-Flash is used 7$\times$ more for chemistry-mix tasks (perhaps benefiting from its large context), while GPT-5 dominates boiling tasks (3$\times$). GPT-OSS-120B specializes in identification tasks (lifespan, melt). This specialization emerges purely from learning, without explicit task-model assignments.

\paragraph{Q-value dynamics during switching.}
When the router switches models, Q-values consistently improve: the average Q-delta is +0.032 for error-triggered switches and +0.046 for non-error switches. Notably, 100\% of switches have positive Q-delta, indicating the router confidently believes switching improves expected outcomes.

\subsection{Ablation Studies}

\begin{table}[t]
\centering
\small
\begin{tabular}{lc}
\toprule
\textbf{Variant} & \textbf{SW Score} \\
\midrule
Full model & -- \\
\midrule
w/o residual embeddings & -- \\
w/o attribute features & -- \\
Ridge instead of MLP & -- \\
3-model pool only & -- \\
\bottomrule
\end{tabular}
\caption{Ablation study on ScienceWorld test set.}
\label{tab:ablation}
\end{table}


Table~\ref{tab:ablation} reports ablations on ScienceWorld; results are omitted here pending final runs.
